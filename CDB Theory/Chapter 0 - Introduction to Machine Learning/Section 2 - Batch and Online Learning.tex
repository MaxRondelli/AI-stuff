\section{Batch and Online learning}
The ability of machine learning systems to learng gradually from an incoming data stream is another factor used to categorize them.

\subsection{Batch learning}
The system cannot learn incrementally while using batch learning. It needs to be trained with all of the information at present. This is usually done offline because it will require a significant amount of time and computing power. After the system has been trained, it is put into production and continues to function without learning. 
Unfortunately, because the world keeps changing while the model stays the same, a model's performance gradually deteriorates over time.
Regularly retraining the model on current data is the solution. The frequency of that depends on the use case: if the model is used to classify images of dogs and cats, it will likely decay very slowly; however, if the model is used to analyze rapidly evolving systems, such as financial market predictions, it is likely to decay quickly. 
A new version of the system must be trained on the entire dataset in order for a batch learning system to learn about new data. The old model must then be replaced with the new one. 
In all cases, using algorithms with incremental learning capabilities is a better solution. 

\subsection{Online learning}
Online learning requires adding data instances to the system one after the other, either individually or in small groups known as mini-batches, in order to train the system incrementally. 
Because each learning phase is quick and cheap, the system can absorb new data as it comes in. 
For systems like the previously mentioned one that must adjust to change very quickly, online learning is helpful.
It is also a wise choice if computing resources are limited.
Furthermore, models can be trained on enormous datasets that are too large to fit in a single machine's main memory using online learning algorithms. The algorithm loads a portion of the data, uses that portion of the data for a training step, and keeps doing this until it has used all of the data.
