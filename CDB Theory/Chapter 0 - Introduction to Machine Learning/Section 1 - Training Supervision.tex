\section{Training Supervision}
Machine learning systems can be classified according the amount and type of supervision they get during the training. 
There are many categoreis:
\begin{itemize}
    \item Supervised learning
    \item Unsupervised learning
    \item Semi-supervised learning
    \item Self-supervised learning
    \item Reinforcement learning
\end{itemize}

Let's see those categoreis in more details. 

\subsection{Supervised learning}
In supervised learning, the trainig set you feed to the algorithm includes the desired solution, which is called label. A typical supervised learning task is classification. 
Another task is to predict a target numeric value, such as the price of a car, given a set of features, like mileage, age and brand. This sort of task is called regression. 
To train the system, you need to give it many examples of cars, including both their features and their targets.

\subsection{Unsupervided learning}
In unsupervised learning, the training data is unlabeled. The system tries to learn without a teacher. Clustering is an example of unsupervised learning.

\subsection{Semi-supervised learning}
Since labeling data is usually time-comsuming and costly, you will have plenty of unlabeled instances and few labeled instances. Some algorithm can deal with data that's partially labeled. 
This is called semi-supervised learning. Those algorithms are a combination of unsupervised and supervised learning.

\subsection{Self-supervised learning}
This approach, called self-supervised learning, involves generating a fully labeled dataset from a fully unlabeled one. Once the whole dataset is labeled, any supervised learning algorithm can be used.
If you have a large dataset of unlabeled images, you can randomly mask a small part of each image and then train a model to recover the original image. 
During training, the masked images are used as the inputs to the model, and the original images are used as the labels. 

\subsection{Reinforcement learning}
The mechanism of reinforcement learning is entirely different. The learning system, called an agent, can observe the environment, select and perform actions,
and get rewards in return. It can get penalties in the form of negative rewards if the action made is wrong. It must then learn by itself
what is the best strategy, called a policy, to get the most reward over time. A policy defines what action the agent should choose when it is in a given situation. 