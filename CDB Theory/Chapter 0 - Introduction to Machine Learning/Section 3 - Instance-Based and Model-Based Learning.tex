\section{Instance-Based and Model-Based Learning}
The way machine learning systems generalize can also be used to classify them. Prediction is the main goal of most machine learning tasks.
This implies that the system must be able to accurately predict instances it has never seen before given a sufficient number of training examples. 
While useful, having a strong performance metric on the training set is insufficient. Getting good results on fresh instances is the real goal. 
There are two way to generalization: instance-based learning and model-based learing. 

\subsection{Instance-based learning}
The system learns from examples. It applies generalization to new cases by comparing them to the learned examples using a similarity measure. 
It builds hypotheses based on the training cases in question. This implies that the complexity of the hypothesis may increase as data do. 
The capacity of instance-based learning to modify its model in response to data that hasn't been seen before gives it an advantage over other machine learning techniques. 
The k-nearest neighbors algorithm is one example of an instance-based learning algorithm.

\subsection{Model-based learning}
Creating a model of a set of examples and using it to generate predictions is another method for generalizing from a set of examples. We refer to this as model-based learning.
We can divide the pipeline in four steps: 
\begin{enumerate}
    \item Study the data
    \item Select the model
    \item Train the model on the training data. The learning algorithm searchs for the model parameter values that minimize a cost function. 
    A cost function is a measure of how well a machine learning model performs by quantifying the difference between predicted and actual outputs. Its goal is to be minimized by adjusting the model's parameters during training.
    \item Finally, the model can make predictions on new cases. Hoping that htis model will generalize well.
\end{enumerate}

This is what a typical machine learing project looks like.