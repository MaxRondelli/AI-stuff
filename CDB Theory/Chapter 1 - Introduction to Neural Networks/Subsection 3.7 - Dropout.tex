\subsection{Dropout}
Dropout, Figure \ref{fig:Dropout}, is one of the most popular regularization techniques for deep neural networks. It was proposed in a paper - by Geoffry Hinton et al. in 2012. 
It is a relatively straightforward algorithm: at each training step, each neuron (including input neurons but never output neurons) has a probability $p$ of being temporarily "dropped out," which means it will be completely ignored during this training phase but may be active during the next. This means that their activations are set to zero, and their incoming and outgoing connections are ignored. This has the effect of reducing the number of parameters in the network and making it more difficult for the network to memorize the training data. The dropout rate, also known as the hyperparameter p, is normally set between 10\% and 50\%; in recurrent neural networks, it is more likely to be between 20\% and 30\%. 

\begin{figure}[h!]
    \centering
    \begin{tikzpicture}[node/.style={circle, draw, thick}, scale = 0.85, transform shape]

      \foreach \y in {1,...,5}{
          \node[node] (i\y) at (0,\nodesep*\y) {};
          \node[node, right=\layersep of i\y] (h1\y) {};
          \node[node, right=\layersep of h1\y] (h2\y) {};
        }
    
      \node[node, right=\layersep of h22] (o1) {};
      \node[node, right=\layersep of h24] (o2) {};
    
      \foreach \source in {1,...,5}
      \foreach \dest in {1,...,5}{
          \path[-stealth, thick] (i\source) edge (h1\dest);
          \path[-stealth, thick] (h1\source) edge (h2\dest);
        }
      \foreach \source in {1,...,5}
      \foreach \dest in {1,2}
      \draw[-stealth, thick] (h2\source) -- (o\dest);
    
      \draw[-stealth, thick] (7.5,3*\nodesep) -- node[above,font=\Large\bfseries] {dropout} (9.5, 3*\nodesep);
    
      % Boundary
    
      \foreach \y in {1,...,5}
      \node[node, right=15em of h2\y] (di\y) {};
    
      \node[red,font=\huge] at (di1) {$\times$};
      \node[red,font=\huge] at (di3) {$\times$};
    
      \foreach \y in {1,...,5}
      \node[node, right=\layersep of di\y] (dh1\y) {};
    
      \node[red,font=\huge] at (dh11) {$\times$};
      \node[red,font=\huge] at (dh13) {$\times$};
      \node[red,font=\huge] at (dh14) {$\times$};
    
      \foreach \y in {1,...,5}
      \node[node, right=\layersep of dh1\y] (dh2\y) {};
    
      \node[red,font=\huge] at (dh22) {$\times$};
      \node[red,font=\huge] at (dh24) {$\times$};
    
      \node[node, right=\layersep of dh22] (do1) {};
      \node[node, right=\layersep of dh24] (do2) {};
    
      \foreach \source in {2,4,5}
      \foreach \dest in {2,5}
      \draw[-stealth, thick] (di\source) -- (dh1\dest);
    
      \foreach \source in {2,5}
      \foreach \dest in {1,3,5}
      \draw[-stealth, thick] (dh1\source) -- (dh2\dest);
    
      \foreach \source in {1,3,5}
      \foreach \dest in {1,2}
      \draw[-stealth, thick] (dh2\source) -- (do\dest);
    
    \end{tikzpicture}
    \caption{Dropout Scheme - A regularization technique used in neural networks to prevent overfitting by randomly dropping out some of the neurons in the network during training.}
    \label{fig:Dropout}
\end{figure}


For example, if the dropout rate is 0.5, then during each training iteration, on average, half of the units in the network will be dropped out. This results in a different, randomly perturbed network architecture at each training iteration, which helps prevent overfitting. The dropout rate can be tuned through experiments to find the optimal value for a given problem and network architecture. In general, a dropout rate of 0.5 is a good starting point, but the optimal value will depend on the specifics of the problem and the network architecture.