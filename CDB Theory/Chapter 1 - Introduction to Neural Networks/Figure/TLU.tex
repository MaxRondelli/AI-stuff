\begin{figure}[h!]
    \centering
    \begin{tikzpicture}[
        % define styles    
        init/.style={ 
             draw, 
             circle, 
             inner sep=2pt,
             font=\Huge,
             join = by -latex
        },
        squa/.style={ 
            font=\Large,
            join = by -latex
        }
    ]
        
        % Top chain x1 to w1
        \begin{scope}[start chain=1]
            \node[on chain=1] at (0,1.5cm)  (x1) {$x_1$};
            \node[on chain=1,join=by o-latex] (w1) {$w_1$};
        \end{scope}
        
        % Middle chain x2 to output
        \begin{scope}[start chain=2]
            \node[on chain=2] (x2) {$x_2$};
            \node[on chain=2,join=by o-latex] {$w_2$};
            \node[on chain=2,init] (sigma) {$\displaystyle\Sigma$};
            \node[on chain=2,squa,label=above:{\parbox{2cm}{\centering Activation\\ function}}]   {$f_{act}$};
            \node[on chain=2,squa,label=above:Output,join=by -latex] {$y_{out}$};
        \end{scope}
        
        % Bottom chain x3 to w3
        \begin{scope}[start chain=3]
            \node[on chain=3] at (0,-1.5cm) 
            (x3) {$x_3$};
            \node[on chain=3,label=below:Weights,join=by o-latex]
            (w3) {$w_3$};
        \end{scope}
        
        % Bias
        \node[label=above:\parbox{2cm}{\centering Bias \\ $b$}] at (sigma|-w1) (b) {};
        
        % Arrows joining w1, w3 and b to sigma
        \draw[-latex] (w1) -- (sigma);
        \draw[-latex] (w3) -- (sigma);
        \draw[o-latex] (b) -- (sigma);
        
        % left hand side brace
        \draw[decorate,decoration={brace,mirror}] (x1.north west) -- node[left=10pt] {Inputs} (x3.south west);
        
    \end{tikzpicture}
    
    \caption{The Threshold Logic Unit (TLU) - A fundamental building block of artificial neural networks, implementing a linear decision boundary to classify input data.}
    \label{fig:TLU Scheme}
\end{figure}