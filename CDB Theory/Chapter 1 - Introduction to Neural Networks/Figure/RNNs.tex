\begin{figure}
    \centering
    \begin{tikzpicture}[item/.style={circle,draw,thick,align=center}, itemc/.style={item,on chain,join}]
         \begin{scope}[start chain=going right,nodes=itemc,every join/.style={-latex,very thick},local bounding box=chain]
         \path node (A0) {$A$} node (A1) {$A$} node (A2) {$A$} node[xshift=2em] (At) {$A$};
         \end{scope}
         \node[left=1em of chain,scale=2] (eq) {$=$};
         \node[left=2em of eq,item] (AL) {$A$};
         \path (AL.west) ++ (-1em,2em) coordinate (aux);
         \draw[very thick,-latex,rounded corners] (AL.east) -| ++ (1em,2em) -- (aux) 
         |- (AL.west);
         \foreach \X in {0,1,2,t} 
         {\draw[very thick,-latex] (A\X.north) -- ++ (0,2em)
         node[above,item,fill=gray!10] (h\X) {$\hat{y}_{(\X)}$};
         \draw[very thick,latex-] (A\X.south) -- ++ (0,-2em)
         node[below,item,fill=gray!10] (x\X) {$x_\X$};}
         \draw[white,line width=0.8ex] (AL.north) -- ++ (0,1.9em);
         \draw[very thick,-latex] (AL.north) -- ++ (0,2em)
         node[above,item,fill=gray!10] {$\hat{h}_{(t)}$};
         \draw[very thick,latex-] (AL.south) -- ++ (0,-2em)
         node[below,item,fill=gray!10] {$x_t$};
         \path (x2) -- (xt) node[midway,scale=2,font=\bfseries] {\dots};
    \end{tikzpicture}
    \caption{A Recurrent Neuron (left) Unrolled through Time (right) - Recurrent neural networks (RNNs) are used to model sequences of data, with each neuron in the network receiving input not only from the current time step but also from the previous time step.}
    \label{fig:RNNs}
\end{figure}