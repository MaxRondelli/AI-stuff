\section{Recurrent Neural Networks}
A class of nets called recurrent neural networks (RNNs) is capable of foreseeing the future. RNNs are capable of analyzing a variety of time series data, like the number of daily visitors to your website, the local hourly temperature, and more. An RNN can forecast the future using its knowledge of past patterns in the data, presuming of course that those patterns will continue to exist. Similar in appearance to a feedforward neural network, a recurrent neural network also includes connections pointing backward.

\hspace{1cm}

Let's discuss the simplest possible RNN, which consists of a single neuron taking inputs, producing output, and sending that output back to the neuron that received it, Figure \ref{fig:RNNs} (left). This recurrent neuron receives its own output from the previous time step $\hat{y}_{(t-1)}$ and the inputs $x_{(t)}$ at each time step $t$. At the first time step, the output is set at $0$, since there is no output at the previous time step. 
\begin{figure}
    \centering
    \begin{tikzpicture}[item/.style={circle,draw,thick,align=center}, itemc/.style={item,on chain,join}]
         \begin{scope}[start chain=going right,nodes=itemc,every join/.style={-latex,very thick},local bounding box=chain]
         \path node (A0) {$A$} node (A1) {$A$} node (A2) {$A$} node[xshift=2em] (At) {$A$};
         \end{scope}
         \node[left=1em of chain,scale=2] (eq) {$=$};
         \node[left=2em of eq,item] (AL) {$A$};
         \path (AL.west) ++ (-1em,2em) coordinate (aux);
         \draw[very thick,-latex,rounded corners] (AL.east) -| ++ (1em,2em) -- (aux) 
         |- (AL.west);
         \foreach \X in {0,1,2,t} 
         {\draw[very thick,-latex] (A\X.north) -- ++ (0,2em)
         node[above,item,fill=gray!10] (h\X) {$\hat{y}_{(\X)}$};
         \draw[very thick,latex-] (A\X.south) -- ++ (0,-2em)
         node[below,item,fill=gray!10] (x\X) {$x_\X$};}
         \draw[white,line width=0.8ex] (AL.north) -- ++ (0,1.9em);
         \draw[very thick,-latex] (AL.north) -- ++ (0,2em)
         node[above,item,fill=gray!10] {$\hat{h}_{(t)}$};
         \draw[very thick,latex-] (AL.south) -- ++ (0,-2em)
         node[below,item,fill=gray!10] {$x_t$};
         \path (x2) -- (xt) node[midway,scale=2,font=\bfseries] {\dots};
    \end{tikzpicture}
    \caption{A Recurrent Neuron (left) Unrolled through Time (right) - Recurrent neural networks (RNNs) are used to model sequences of data, with each neuron in the network receiving input not only from the current time step but also from the previous time step.}
    \label{fig:RNNs}
\end{figure}
This little network can be shown in relation to the time axis, Figure \ref{fig:RNNs} (right). "Unrolling the network through time" is what is meant by this. Each neuron receives the output vector from the previous time step, $\hat{y}_{(t-1)}$, as well as the input vector $x_{(t)}$, at each time step $t$. As you can see, now inputs and outputs are vectors. One set of weights is for the inputs $x_{(t)}$, and the other is for the outputs of the previous time step $\hat{y}_{(t-1)}$ for each recurrent neuron. These weight vectors will be abbreviated $w_x$ and $w_{\hat{y}}$. We can organize all the weight vectors into two weight matrices, $W_x$ and $W_{\hat{y}}$, if we think about the entire recurrent layer rather than just one recurrent neuron.
The output vector of the entire recurrent layer can then be calculated in a similar way to what one might anticipate.

\begin{equation}
\hat{y}_{(t)} = \sigma \left(W_{x}^{T} x_{(t)} + w_{\hat{y}}^{T} \hat{y}_{(t-1)} + b \right)
\end{equation}

Like feedforward neural networks, by putting all the inputs at time step t into an input matrix X, we can compute the output of a recurrent layer in one single step for an entire mini-batch.
    
\begin{equation}
    \begin{split}
    \hat{Y}_{(t)} & = \sigma \left(X_{(t)} W_x + \hat{Y}_{(t-1)} W_{\hat{y}} + b \right) \\
    & = \sigma \left(\begin{bmatrix} X_{(t)} & \hat{Y}_{(t-1)} \end{bmatrix} W + b \right) \text{with } W = \begin{bmatrix} 
    W_{x} \\
    W_{\hat{y}} 
    \end{bmatrix}
    \end{split}
    \label{eq:2_12}
\end{equation}

In above equation, \eqref{eq:2_12}, we can see:
\begin{itemize}
    \item $\hat{Y}_{(t)}$ is an \text{\textit{m} $\times$ $n_{\text{neurons}}$} matrix containing the layer's output at time step $t$ for each instance in the mini batch.
    \item $X_{(t)}$ is an \text{\textit{m} $\times$ $n_{\text{inputs}}$} matrix containing the inputs for all instances. 
    \item $W_{x}$ is an \text{$n_{\text{inputs}} \times n_{\text{neurons}}$} matrix containing the connection weights for the inputs of the current time step.
    \item $W_{\hat{y}}$ is an \text{$n_{\text{neurons}}$ $\times$ $n_{\text{neurons}}$} matrix containing the connection weights for the outputs of the previous time step.
    \item $b$ is a vector of size $n_{\text{neurons}}$ containing each neuron's bias term.
    \item The weight matrices $W_{x}$ and $W_{\hat{y}}$ are concatenated vertically into a single weight matrix $W$.
    \item The notation $\begin{bmatrix} X_{(t)} & \hat{Y}_{(t-1)} \end{bmatrix}$ represents the horizontal concatenation of the matrices $X_{(t)}$ and $\hat{Y}_{(t-1)}$.
\end{itemize}

Notice that $\hat{Y}_{(t)}$ is a function of $X_{(t)}$ and $\hat{Y}_{(t-1)}$, which is a function of $X_{(t-1)}$ and $\hat{Y}_{(t-2)}$, which is a function of $X_{(t-2)}$ and $\hat{Y}_{(t-3)}$, and so on. This makes $\hat{Y}_{(t)}$ a function of all the inputs since time $t = 0$ (that is $X_{(0)}, X_{(1)}, X_{(2)}, ...., X_{(t)}$). At the first time step, $t = 0$, there are no previous outputs, so they are assumed to be all zeros.

\subsection{How to train RNNs}
You could say that a recurrent neuron has a form of memory because its output at a given time step $t$ is a function of all its inputs from earlier time steps. A \textit{memory cell} is a component of a neural network that keeps a certain state over successive time steps.
The state of a cell at time step $t$, represented by the symbol $h_{(t)}$, is a function of some inputs at that time step and its state at the previous time step.
So, we can say $h_{(t)} = f\left(x_{(t)}, h_{(t-1)}\right)$. The previous state and the current inputs are functions of the output at time step $t$, indicated as $\hat{y}_{(t)}$.

\hspace{1cm}

An RNN can accept a series of inputs and generate different sequences:
\begin{itemize}
    \item \textit{Sequence-to-sequence network:} it takes a sequence of inputs and produces a sequence of outputs at each time step $t$.
    \item \textit{Sequence-to-vector network:} it takes a sequence of inputs, and you can consider only some outputs. For example, if you have 5 inputs, you might want only the last output, so you can ignore all the previous outputs.
    \item \textit{Vector-to-sequence network:} The input sequence is a vector that you pass into the network at each time step and let it output a sequence. 
    \item \textit{Encoder-decoder network:} This network is mostly used for translations. You pass a sentence in one language, and the output will be translated in another language. 
\end{itemize}

The idea is to unroll an RNN over time before using traditional backpropagation to train it. The term \textit{backpropagation over time} (BPTT) refers to this technique. The network is initially passed forward after it has been unrolled. After that, a loss function is used to evaluate the output sequence. 
\begin{equation}
    L(Y_{(0)}, Y_{(1)}, ...,Y_{(T)}; \hat{Y}_{(0)}, \hat{Y}_{(1)}, ...,\hat{Y}_{(T)})
\end{equation}

where $Y_{(i)}$ is the $i^{th}$ output, $\hat{Y}_{(i)}$ is the $i^{th}$ prediction and $T$ is the max time step. 
For example, if we think about \textit{sequence-to-vector network}, we want to compute only just the last two outputs of the network, ignoring the first three outputs. It means that the loss function isn't computed on all outputs, but just on the last two.

The unrolled network then propagates the gradients of that loss function backward. The gradients only pass through the outputs $\hat{Y}_{(3)}$ and $\hat{Y}_{(4)}$, since in the example the outputs $\hat{Y}_{(0)}$, $\hat{Y}_{(1)}$ and $\hat{Y}_{(2)}$ are not used to calculate the loss. Thusly, because $W$ and $b$ are identical parameters at every time step, their gradients will be changed numerous times during backprop. The parameters can be updated using a gradient descent step using BPTT when the backward phase is finished, and all the gradients have been computed.
This is how RNN training is made. 

\subsection{Long short-term memory (LSTM)}
In 1997, Seep Hochreiter and Jürgen Schmidhuber proposed the "Long Short-Term Memory" (LSTM) cell, which was progressively improved over time by other researchers \parencite{hochreiter1997long} . If the LSTM cell is viewed as a black box, it can be used in a similar way to that of a basic cell but will perform much better. Training will converge more quickly and find longer-term patterns in the data.

How do LSTM cells work? Figure \ref{fig:LSTM_cell} represents its architecture. The LSTM cell appears just like a standard cell from the outside, except for the fact that its state has been divided into two vectors, $h_{(t)}$ and $c_{(t)}$. The short-term state is represented by $h_{(t)}$, and the long-term state is represented by $c_{(t)}$.

The main concept is that the network can learn what to read from it, and what to discard or store in the long-term state. You can see that as the long-term state $c_{(t-1)}$ moves from left to right throughout the network, it first uses a forget gate to delete some memories before adding some new ones using an addition operation that includes memories that were chosen by an input gate. Without any further change, the result $c_{(t-1)}$ is sent out directly. Several memories are added, and some are deleted at each time step. The long-term state is copied and then passed via the tanh function following the addition operation. The output gate then filters the outcome. This produces the short-term state $h_{(t)}$.

\begin{figure}[b!]
    \centering
    \begin{tikzpicture}[
        % GLOBAL CFG
        font=\sf \scriptsize,
        >=LaTeX,
        % Styles
        cell/.style={% For the main box
            rectangle, 
            rounded corners=5mm, 
            draw,
            very thick,
            },
        operator/.style={%For operators like +  and  x
            circle,
            draw,
            inner sep=-0.5pt,
            minimum height =.2cm,
            },
        function/.style={%For functions
            ellipse,
            draw,
            inner sep=1pt
            },
        ct/.style={% For external inputs and outputs
            circle,
            draw,
            line width = .75pt,
            minimum width=1cm,
            inner sep=1pt,
            },
        gt/.style={% For internal inputs
            rectangle,
            draw,
            minimum width=4mm,
            minimum height=3mm,
            inner sep=1pt
            },
        mylabel/.style={% something new that I have learned
            font=\scriptsize\sffamily
            },
        ArrowC1/.style={% Arrows with rounded corners
            rounded corners=.25cm,
            thick,
            },
        ArrowC2/.style={% Arrows with big rounded corners
            rounded corners=.5cm,
            thick,
            },
        ]
    
    %Start drawing the thing...    
        % Draw the cell: 
        \node [cell, minimum height =4cm, minimum width=6cm] at (0,0){} ;
    
        % Draw inputs named ibox#
        \node [gt] (ibox1) at (-2,-0.75) {$\sigma$};
        \node [gt] (ibox2) at (-1.5,-0.75) {$\sigma$};
        \node [gt, minimum width=1cm] (ibox3) at (-0.5,-0.75) {Tanh};
        \node [gt] (ibox4) at (0.5,-0.75) {$\sigma$};
    
       % Draw opérators   named mux# , add# and func#
        \node [operator] (mux1) at (-2,1.5) {$\times$};
        \node [operator] (add1) at (-0.5,1.5) {+};
        \node [operator] (mux2) at (-0.5,0) {$\times$};
        \node [operator] (mux3) at (1.5,0) {$\times$};
        \node [function] (func1) at (1.5,0.75) {Tanh};
    
        % Draw External inputs? named as basis c,h,x
        \node[ct, label={[mylabel]Cell}] (c) at (-4,1.5) {\empt{c}{t-1}};
        \node[ct, label={[mylabel]Hidden}] (h) at (-4,-1.5) {\empt{h}{t-1}};
        \node[ct, label={[mylabel]left:Input}] (x) at (-2.5,-3) {\empt{x}{t}};
    
        % Draw External outputs? named as basis c2,h2,x2
        \node[ct, label={[mylabel]Label1}] (c2) at (4,1.5) {\empt{c}{t}};
        \node[ct, label={[mylabel]Label2}] (h2) at (4,-1.5) {\empt{h}{t}};
        \node[ct, label={[mylabel]left:Label3}] (x2) at (2.5,3) {\empt{\hat{y}}{t}};
    
    % Start connecting all.
        %Intersections and displacements are used. 
        % Drawing arrows    
        \draw [ArrowC1] (c) -- (mux1) -- (add1) -- (c2);
    
        % Inputs
        \draw [ArrowC2] (h) -| (ibox4);
        \draw [ArrowC1] (h -| ibox1)++(-0.5,0) -| (ibox1); 
        \draw [ArrowC1] (h -| ibox2)++(-0.5,0) -| (ibox2);
        \draw [ArrowC1] (h -| ibox3)++(-0.5,0) -| (ibox3);
        \draw [ArrowC1] (x) -- (x |- h)-| (ibox3);
    
        % Internal
        \draw [->, ArrowC2] (ibox1) -- (mux1);
        \draw [->, ArrowC2] (ibox2) |- (mux2);
        \draw [->, ArrowC2] (ibox3) -- (mux2);
        \draw [->, ArrowC2] (ibox4) |- (mux3);
        \draw [->, ArrowC2] (mux2) -- (add1);
        \draw [->, ArrowC1] (add1 -| func1)++(-0.5,0) -| (func1);
        \draw [->, ArrowC2] (func1) -- (mux3);
    
        %Outputs
        \draw [-, ArrowC2] (mux3) |- (h2);
        \draw (c2 -| x2) ++(0,-0.1) coordinate (i1);
        \draw [-, ArrowC2] (h2 -| x2)++(-0.5,0) -| (i1);
        \draw [-, ArrowC2] (i1)++(0,0.2) -- (x2);
    \end{tikzpicture}
    \caption{An LSTM Cell - A type of recurrent neural network cell that can selectively remember or forget information from previous time steps, making it particularly useful for processing sequential data such as text and speech.}
    \label{fig:LSTM_cell}
\end{figure}

Now let's see how gates perform. First, four separate fully connected layers take the current input vector $x_{(t)}$ and the previous short-term state $h_{(t-1)}$. Each one has a specific function:
\begin{itemize}
    \item The layer that outputs $g_{(t)}$ is the primary layer. Its regular functions include processing the inputs for the present $x_{(t)}$ and the past $h_{(t-1)}$ states. The output of this layer is not sent directly outside. Instead, its most fundamental parts are stored in the long-term state. The rest is dropped.
    \item Gate controllers are the three additional layers. The outputs are in the sigmoid activation function range, which is 0 to 1. The outputs from the gate controllers are fed into element-wise multiplication processes; if the output is a 0, the gate is closed; if a 1, the gate is opened. Particularly:
    \begin{itemize}
        \item The \textit{forget gate} $f_{(t)}$: determines which elements of the long-term state should be removed.
        \item The \textit{input gate} $i_{(t)}$: determines which $g_{(t)}$ components go into the long-term state.
        \item The \textit{output gate} $o_{(t)}$: determines which elements of the long-term state should be read and output at this time step, both to $h_{(t)}$ and $y_{(t)}$, 
    \end{itemize}
\end{itemize}

In conclusion, we can say that an LSTM cell can understand how to identify important input, \textit{"input gate"} role, and it can store in a long-term state, preserve and use it whenever it wants, \textit{"forget gate"} role.
There are more variants of the LSTM cell. Let's see now, the most used and important: the \textit{GRU} cell.

\begin{figure}[h!]
    \centering
    \begin{tikzpicture}[>=stealth, scale=1.25]
        %The rectangle :
        \filldraw[rounded corners, opacity=0.5, fill=white](0,0)rectangle(7,5);
        \draw[rounded corners, line width=0.4mm, tgray] (0,0)rectangle(7,5);
        %The connection lines and + x nodes : 
        \draw[thick, tgray, ->] (-1,4.5)node[left] {$h_{(t-1)}$}--(0,4.5);
        \draw[thick, tgray] (0,4.5)--(2,4.5);
        \draw[thick, tgray] (2.5,4.5)--(4.5,4.5);
        \draw[thick, tgray, ->] (5,4.5)--(8,4.5) node[right] {$h_{(t)}$};
        \draw[thick, tgray, ->] (6.5,4.5)--(6.5,6) node[right] {$y_{(t)}$};
        \node[tgray] (o1) at (2.25,4.5) {$\bigotimes$};
        \node[tgray] (o2) at (4.75,4.5) {$\bigoplus$};
        \draw[thick, tgray] (0.5,4.5)--(0.5,0.5);
        \draw[thick, tgray] (0.5,2)--(1,2); 
        \draw[thick, tgray] (1.5,2)--(2,2); 
        \draw[thick, tgray, ->] (1.25,1.35)--(1.25,1.8) node[midway, left] {$r_{(t)}$};
        \draw[thick, tgray, ->] (0.5,0.5)--(1.25,0.5)--(1.25,0.7);
        \draw[thick, tgray, ->] (2.25,1.35)  node[above left] {$z_{(t)}$} --(2.25,4.25);
        \draw[thick, tgray, ->] (2.25,-0.75) node[below] {$x_{(t)}$}--(2.25,0.7);
        \draw[thick, tgray, ->] (2.5,2)--(4.5,2)--(4.5,2.25);
        \draw[thick, tgray, ->] (1.25,0.5)--(2.25,0.7);
        \draw[thick, tgray, ->] (2.25,0.5)--(1.25,0.7);
        \node[tgray] (o3) at (1.25,2) {$\bigotimes$};
        \coordinate(o) at (3,3.5);
        \draw[tgray, thick] (2.25,3)--(o);
        \draw[tgray, thick, ->] (o)--(4.5,3.5);
        \filldraw[tgray] (o)circle(0.15) node[white, scale=0.5] {$1-$};
        \draw[thick, tgray, ->] (4.75,2.85)--(4.75,3.25);
        \draw[thick, tgray, ->]  (4.75,3.75)--(4.75,4.25);
        \node[tgray] at (4.75,3.5) {$\bigotimes$};
        %The small FC boxes
        %Box 1
        \draw[tgray] (1,0.75)rectangle(1.5,1.25); 
        \node[scale=0.85] at (1.25,1) {FC};
        \fill[tgray] (1,1.25)rectangle(1.5,1.3);
        %Box 2
        \draw[tgray] (2,0.75)rectangle(2.5,1.25); 
        \node[scale=0.85] at (2.25,1) {FC};
        \fill[tgray] (2,1.25)rectangle(2.5,1.3);
        %Box 3
        \draw[tgray] (4.25,2.3)rectangle(5.25,2.75);
        \node[scale=0.85] at (4.75,2.525) {FC};
        \fill[tgray] (4.25,2.75)rectangle(5.25,2.8);
    \end{tikzpicture}
    \caption{An GRU Cell - A type of recurrent neural network cell that uses gating mechanisms to control the flow of information through the cell, allowing it to selectively update or retain information from previous time steps.}
    \label{fig:my_label}
\end{figure}

\subsection{Gated Recurrent Unit (GRU)}
In a 2014 paper, Kyunghyun Cho et al. made the suggestion for the Gated Recurrent Unit (GRU) cell \parencite{cho2014learning}.
The GRU cell, which is an LSTM cell simplified, seems to work just as well. The main changes are as follows:

\begin{itemize}
    \item A single vector $h_{(t)}$ is created by combining the two state vectors.
    \item Both the input gate and the forget gate are managed by a single gate controller $z_{(t)}$. The input gate is closed $(1-1 = 0)$ and the forget gate is open $(= 1)$ if the gate controller sends a $1$. The opposite occurs if the output is a $0$. To put it another way, whenever a memory needs to be saved, the area where it will be stored must first be deleted.
    \item The entire state vector is output at each time step; there is no output gate. The main layer $g_{(t)}$ will only see certain portions of the prior state, thanks to a new gate controller $r_{(t)}$.
\end{itemize}

One of the key elements in the success of RNNs is the use of LSTM and GRU cells.

